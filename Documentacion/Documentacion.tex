\title{}
\date{\today}
\documentclass[12pt]{article}

\usepackage[utf8]{inputenc}
\usepackage{appendix}
\usepackage[hidelinks]{hyperref}

\author{Carlos Bergillos, Adrià Cabeza, Roger Vilaseca}

\setcounter{tocdepth}{2}

\begin{document}

\maketitle

\newpage

\tableofcontents

\newpage


\section{Introducción}

\section{Descripción del problema}
En esta práctica nos encargaremos de planificar un proyecto de programación de gran envergadura.
Deberemos repartir un conjunto de tareas a realizar entre los programadores disponibles.

En concreto, disponemos de $T$ tareas de programación, cada una de ellas tiene un grado de dificultad asignado (de 1 a 3), y una tiempo estimado de realización estimado (en horas).

También disponemos de un conjunto $P$ de programadores. Cada uno de ellos tiene asignado un grado de habilidad (de 1 a 3), que nos indica lo mucho o poco que esta capacitado para resolver las tareas.

No queremos asignar a los programadores tareas mucho más dificiles de lo que para ellos estan capacitados. En concreto, a un programador solo le podremos asignar tareas de como mucho una unidad más de dificultad de lo que nos indique su habilidad. En caso de que sea necesario asignar a un programador una tarea más dificil que su capacidad, la duración de la realización de la tarea se verá incrementada en 2 horas.

Además, todas las tareas deberán ser revisadas, para ello, hará falta una nueva tarea adicional. Esta nueva tarea de revisión será de la misma dificultad que la tarea original.
Los programadores tienen tambien asociada una calidad (de 1 a 2). Si la tarea original es realizada por un programador de calidad 1, la nueva tarea de revisión durará 1 hora, si en cambio el programador era de calidad 2, la nueva tarea de revisión durará 2 horas. 
cada programador también tiene asociada una calidad (de 1 a 2)
Para evitar una recursividad infinita, las nuevas tareas de revisión no requerirán a su vez de revisión, y su tiempo de realización no se verá penalizado por la habilidad del programador que la realiza.


\section{Nivel básico}
En esta primera versión....


\subsection{Dominio}
\subsubsection{Variables}
Para la correcta resolución de este problema de planificación hemos visto conveniente trabajar con variables con tipo.
En concreto, hemos necesitado 2 tipos, los cuáles hemos llamado \textbf{programador} y \textbf{tarea}.

\begin{itemize}
  \item \textbf{programador:} Se utilizará para las variables que correspondan a cada programador del conjunto $P$.
  \item \textbf{tarea:} Se utilizará para las variables que correspondan a cada tarea del conjunto $T$.
\end{itemize}

Para las próximas extensiones ya no se requieren más cambios en las variables, esta será la configuración definitiva.

\subsubsection{Funciones}
\begin{itemize}
  \item \textbf{habilidadProgramador:}
  \item \textbf{dificultadTarea:}
\end{itemize}
\subsubsection{Predicados}
\begin{itemize}
  \item \textbf{asignacion:}
  \item \textbf{tareaAsignada:}
\end{itemize}
\subsubsection{Acciones}
\begin{itemize}
  \item \textbf{asignar:}
\end{itemize}

\subsection{Problema}
\subsubsection{Objetos}
\subsubsection{Estado inicial}
\subsubsection{Estado final}

\subsection{Juegos de prueba}


\section{Extensión 1}

\subsection{Dominio}
\subsubsection{Funciones}
\subsubsection{Predicados}
\subsubsection{Acciones}

\subsection{Problema}
\subsubsection{Objetos}
\subsubsection{Estado inicial}
\subsubsection{Estado final}

\subsection{Juegos de prueba}


\section{Extensión 2}

\subsection{Dominio}
\subsubsection{Funciones}
\subsubsection{Predicados}
\subsubsection{Acciones}

\subsection{Problema}
\subsubsection{Objetos}
\subsubsection{Estado inicial}
\subsubsection{Estado final}

\subsection{Juegos de prueba}


\section{Extensión 3}

\subsection{Dominio}
\subsubsection{Funciones}
\subsubsection{Predicados}
\subsubsection{Acciones}

\subsection{Problema}
\subsubsection{Objetos}
\subsubsection{Estado inicial}
\subsubsection{Estado final}

\subsection{Juegos de prueba}


\section{Extensión 4}

\subsection{Dominio}
\subsubsection{Funciones}
\subsubsection{Predicados}
\subsubsection{Acciones}

\subsection{Problema}
\subsubsection{Objetos}
\subsubsection{Estado inicial}
\subsubsection{Estado final}

\subsection{Juegos de prueba}


\section{Conclusiones}


\newpage
\appendix
\appendixpage
\addappheadtotoc

\section{Generador de juegos de prueba}


\end{document}